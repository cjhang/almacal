%                                                                 aa.dem
% AA vers. 9.1, LaTeX class for Astronomy & Astrophysics
% demonstration file
%                                                       (c) EDP Sciences
%-----------------------------------------------------------------------
%
%\documentclass[referee]{aa} % for a referee version
%\documentclass[onecolumn]{aa} % for a paper on 1 column  
%\documentclass[longauth]{aa} % for the long lists of affiliations 
%\documentclass[letter]{aa} % for the letters 
%\documentclass[bibyear]{aa} % if the references are not structured 
%                              according to the author-year natbib style

%
\documentclass{ref/aa}  

%
\usepackage{graphicx}
%%%%%%%%%%%%%%%%%%%%%%%%%%%%%%%%%%%%%%%%
\usepackage{txfonts}
%%%%%%%%%%%%%%%%%%%%%%%%%%%%%%%%%%%%%%%%
%\usepackage[options]{hyperref}
% To add links in your PDF file, use the package "hyperref"
% with options according to your LaTeX or PDFLaTeX drivers.
%
\begin{document} 


   \title{ALMACAL: number counts for faint sub-millimeter galaxies}

   \subtitle{To be simple, to be clear}

   \author{J. Chen
          \inst{1}
          \and
          R. Ivison \inst{1}\fnmsep\thanks{Just to show the usage
          of the elements in the author field}
          }

   \institute{European Southern Observatory (ESO), Karl-Schwarzschild-Strasse 2\\
              \email{jianhang.chen@eso.org}
         \and
             Another institude, ...\\
             \email{user@email.com}
             \thanks{The university of heaven temporarily does not
                     accept e-mails}
             }

   \date{Received **, ****; accepted **, ****}

% \abstract{}{}{}{}{} 
% 5 {} token are mandatory
 
  \abstract
  % context heading (optional)
  % {} leave it empty if necessary  
   {Context of the project}
  % aims heading (mandatory)
   {Aims of this project.}
  % methods heading (mandatory)
   {Methods}
  % results heading (mandatory)
   {Main results}
  % conclusions heading (optional), leave it empty if necessary 
   {}

   \keywords{SMG --
              galaxy evolution   --
              cosmic star formation
               }

   \maketitle
%
%-------------------------------------------------------------------

\section{Introduction}
The number counts and galaxy evolution models.

the SMG has the advantage of negative-K correction, which make their flux density almost constant along a large redshift space (z=1-10)

The Lyman break galaxy have the advantage of redshit, but they can be observed only if it is bright enough, which makes them biased to the most brightest galaxy population.

\citep{Oteo2016a}
%--------------------------------------------------------------------
\section{AMLACAL}

%-------------------------------------- Two column figure (place early!)
   \begin{figure*}
   \centering
   %%%\includegraphics{empty.eps}
   %%%\includegraphics{empty.eps}
   %%%\includegraphics{empty.eps}
   \caption{Large picture}%
    \end{figure*}
%

\section{Conclusions}

   \begin{enumerate}
      \item conlusion1
   \end{enumerate}

\begin{acknowledgements}
      Part of this work was supported by the German
      \emph{Deut\-sche For\-schungs\-ge\-mein\-schaft, DFG\/} project
      number Ts~17/2--1.
\end{acknowledgements}


\bibliographystyle{ref/aa} % style aa.bst
\bibliography{ref/almacal.bib} % your references Yourfile.bib


%
\end{document}

%

%-------------------------------------------------------------
%                   For appendices and landscape, large table:
%                    in the preamble, use: \usepackage{lscape}
%-------------------------------------------------------------

\begin{appendix} %First appendix
%
\longtab[1]{
\begin{landscape}
\begin{longtable}{lrcrrrrrrrrl}
...
\end{longtable}
\end{landscape}
}% End longtab
\end{appendix}

%%%% End of aa.dem
